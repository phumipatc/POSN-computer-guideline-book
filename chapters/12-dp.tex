\chapter{การโปรแกรมเชิงพลวัต (Dynamic Programming)}

Dynamic Programming เป็นเทคนิคที่ใช้ในการแก้ปัญหาเพื่อหาคำตอบคล้ายๆ กับ Divide \& Conquer คือทำการแบ่งปัญหาออกเป็นปัญหาย่อยๆ เพื่อนำคำตอบที่ได้มาช่วยให้การแก้ปัญหาใหญ่ง่ายขึ้น ในบทนี้จะกล่าวถึง Dynamic Programming เบื้องต้นที่ควรรู้จัก

\section{Quick Sum}

\begin{problem}
มีอาเรย์ $a$ ที่เก็บจำนวนเต็มทั้งหมด $n$ ตัวอยู่ โดยเริ่มที่ $a_1$ ไปจนถึง $a_n$ ต้องการหาผลรวมของ $a_l + a_{l+1} + a_{l+2} + ... + a_r$ เมื่อ $1\leq l,r \leq n$
\end{problem}

วิธีแก้ปัญหานี้สามารถทำได้โดยง่าย โดยใช้วิธี Brute Force ซึ่งใช้เวลาทำงาน $O(n)$ ต่อการหาผลรวม 1 ครั้ง หากต้องการหาผลรวมทั้งสิ้น $m$  ครั้งจะต้องใช้เวลาการทำงาน $O(nm)$

\begin{lstlisting}
int sum(int l, int r){
	int res = 0;
	for (int i = l; i <= r; i++)
		res += a[i];
	return res;
}
\end{lstlisting}

แต่หากใช้เทคนิค Quick Sum การหาผลรวม 1 ครั้งจะสามารถลดเวลาการทำงานได้เป็น $O(1)$ ได้ทันที

หลักการทำงานของ Quick Sum คือ กำหนดอาเรย์ $qs$ ขึ้นมาโดยมีนิยามว่า $qs_i$ จะเก็บผลรวมตั้งแต่ $a_1$ จนถึง $a_i$ ดังนั้นหากต้องการผลรวมตั้งแต่ $a_l$ จนถึง $a_r$  จึงสามารถหาได้ทันทีด้วย $qs_r - qs_{l-1}$

\begin{lstlisting}
void prefixSum(int n){
	for (int i = l; i <= r; i++)
		qs[i] = qs[i - 1] + a[i];
}
int quickSum(int l, int r){
	return qs[r] - qs[l - 1];
}
\end{lstlisting}

\section{ปัญหาถุงกระสอบ (Knapsack Problem)}

\begin{question}
มีสิ่งของอยู่ $n$ ชิ้น แต่ละชิ้นมีมูลค่า $v_i$ และมีนำ้แหนัก $w_i$ หากต้องการนำของใส่ในถุงกระสอบที่รับนำ้หนักได้ไม่เกิน $m$ จะมีมูลค่ารวมสูงสุดเท่าใด
\end{question}

