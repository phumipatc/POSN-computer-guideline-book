\chapter{Intro to C/C++}
ในหัวข้อนี้ จะเป็นการแนะนำเกี่ยวกับการเขียนโปรแกรมภาษา C และ C++ ในเบื้องต้น เพื่อเป็นพื้นฐานสำหรับการเขียนโปรแกรมที่ซับซ้อนในหัวข้ออื่นๆ ต่อไป
\section{Syntax}
\begin{lstlisting}
#include <stdio.h>

int main() {
  return 0;
}
\end{lstlisting}
โปรแกรมด้านบนนี้เป็นส่วนของโปรแกรมที่ต้องมีสำหรับการเขียนโปรแกรมภาษา C เพื่อให้การทำงานของโปรแกรมเป็นไปได้อย่างปกติ

\textbf{บรรทัดที่ 1:} \textbf{\texttt{\#include}} คือการนำ header file library  เข้ามาใช้ในโปรแกรม โดย header files มีหน้าที่เพิ่มฟังก์ชั่นการทำงานของโปรแกรมให้สามารถทำงานได้ตามวัตุประสงค์ของเรา ในที่นี้ ``stdio.h'' เป็นชื่อที่ย่อมาจาก ``Standard Input Output" ซึ่งเป็น header สำคัญในการรับข้อมูลนำเข้าและส่งออก นอกจากนี้ยังมี header files ที่สำคัญอีกมากมาย อาทิเช่น ``math.h'' ที่มีฟังก์ชั่นเกี่ยวกับคณิตศาสตร์ต่างๆ , ``ctype.h'' ที่มีฟังก์ชั่นเกี่ยวกับตัวอักษรต่างๆ เป็นต้น

\textbf{บรรทัดที่ 3:} เป็นการประกาศฟังก์ชั่นหลักที่ใช้ในการทำงานของโปรแกรม ทุกครั้งที่โปรแกรมเริ่มทำงานจะเริ่มทำงานที่ฟังก์ชั่น main นี้ก่อนเสมอ

\textbf{บรรทัดที่ 4:} คำสั่ง \textbf{\texttt{return 0;}} เป็นคำสั่งเพื่อจบการทำงานของโปรแกรม โดย 0 เป็นรหัสคำสั่งที่โปรแกรมส่งออกให้ทราบว่าการทำงานของโปรแกรมนี้มีความผิดพลาดเป็น 0 (ไม่มี error)

\noindent\textbf{หมายเหตุ:} จะสังเกตเห็นได้ว่าคำสั่งของภาษา C/C++ จะลงท้ายด้วย semicolon (\textbf{\texttt{;}}) เสมอ และขอบเขตการทำงานจะถูกระบุด้วยเครื่องหมายปีกกา (\textbf{\texttt{\{ \}}})

\newpage
\begin{lstlisting}
#include <bits/stdc++.h>
using namespace std;
int main() {
  return 0;
}
\end{lstlisting}
โปรแกรมด้านบนนี้เป็นส่วนของโปรแกรมภาษา C++ โดยสังเกตว่าจะคล้ายกับภาษา C มาก แตกต่างกันเพียงบรรทัดที่ 1 และมีบรรทัดที่ 2 เพิ่มมา

\textbf{บรรทัดที่ 1:} ``bits/stdc++.h" เป็น header file ของภาษา C++ ที่รวมหลาย header files ที่เป็นประโยชน์เข้าไว้ด้วยกัน เพื่อเพิ่มความสะดวกของผู้ใช้ (ไม่ต้องทำการ \textbf{\texttt{\#include}} หลายๆ รอบ)

\textbf{บรรทัดที่ 2:} \textbf{\texttt{using namespace std;}} เป็นคำสั่งเพื่อละการเขียน "\textbf{\texttt{std::}}" หน้าบางคำสั่ง เพื่อให้สามารถเขียนโปรแกรมได้รวดเร็วยิ่งขึ้น

\subsection{Comments}
การเขียนคำอธิบายในภาษา C/C++ สามารถทำได้ 2 วิธี
\begin{enumerate}
\item Single-line comments เป็นการเขียนคำอธิบายแบบบรรทัดเดียว ใช้เครื่องหมาย \textbf{\texttt{//}} ไว้หน้าข้อความที่ต้องการให้กลายเป็นคำอธิบาย และไม่ถูก compile 
\begin{lstlisting}
// This is a single-line comment
#include<bits/stdc++.h>	//Can be used after each line too 
\end{lstlisting}
\item Multi-line comments เป็นการเขียนคำอธิบายแบบหลายบรรทัด ใช้เครื่องหมายเริ่ม \textbf{\texttt{/*}} และเครื่องหมายจบ \textbf{\texttt{*/}} โดยข้อความที่อยู่ระหว่างเครื่องหมายทั้ง 2 จะกลายเป็นคำอธิบายที่ไม่ถูก compile
\begin{lstlisting}
/*
This is the first line of multi-line comment
This is the second line of multi-line comment
*/
\end{lstlisting}
\end{enumerate}

\section{Variables}
ในภาษา C/C++ มีตัวแปรที่เก็บข้อมูลได้หลากหลายประเภทด้วยกัน
\subsubsection{Basic Data Types}
\begin{center}
\begin{tabular}{||c|c|c||}
\hline
\textbf{Data type} & \textbf{Size} & \textbf{Description} \\
\hline
\texttt{boolean} & 1 byte & เก็บค่าความจริง (true / false) \\
\texttt{char} & 1 byte & เก็บตัวอักขระ 1 ตัว \\
\texttt{int} & 2 or 4 bytes & เก็บจำนวนเต็ม \\
\texttt{double} & 8 bytes & เก็บจำนวนทศนิยม ความแม่นยำไม่เกิน 15 หลักทศนิยม \\
\hline
\end{tabular}
\end{center}
นอกจากนี้ ยังมีการนำ \texttt{char} มาประยุกต์เป็น array of char (\texttt{char[]}) เพื่อให้สามารถเก็บชุดอักขระ (String) ได้
\subsection{Variables Declaration}
การประกาศตัวแปรสามารถทำได้โดยการระบุประเภทตัวแปร ชื่อตัวแปร เป็นเบื้องต้น ซึ่งสามารถประกาศได้ 2 วิธี ได้แก่
\begin{enumerate}
\item การประกาศตัวแปรพร้อมระบุค่า
\begin{lstlisting}
int firstNumber = 10;
\end{lstlisting}
\item การประกาศตัวแปรแล้วระบุค่าภายหลัง
\begin{lstlisting}
double secondNumber;
secondNumber = 10.2;
\end{lstlisting}
\end{enumerate}
\textbf{หมายเหตุ:} การระบุค่าใหม่ไปยังตัวแปรที่มีค่าอยู่แล้ว จะเป็นการเขียนค่าใหม่ทับลงไป ค่าเก่าจะหายไป
\begin{lstlisting}
int thirdNumber = 10;
thirdNumber = 15;
//thirdNumber's current value is 15
\end{lstlisting}

นอกจากนี้ การประกาศตัวแปรประเภทเดียวกันหลายตัวแปรสามารถทำได้ในบรรทัดเดียวกัน โดยใช้เครื่องหมาย comma (\textbf{\texttt{,}}) ในการคั่นระหว่างตัวแปร
\begin{lstlisting}
int w = 1, x = 2;
\end{lstlisting}

การระบุค่าเดียวกันให้กับตัวแปรหลายตัวแปรก็สามารถทำได้ในบรรทัดเดียวกันได้
\begin{lstlisting}
int y, z;
y = z = 3;
/*
same result as
y = 3;
z = 3;
*/
\end{lstlisting}

การประกาศตัวแปรค่าคงที่สามารถทำได้คล้ายกับการประกาศตัวแปรทั่วไป เพียงแค่เพิ่ม \textbf{\texttt{const}} ไปข้างหน้าประเภทตัวแปรนั้น โดยตัวแปรค่าคงที่จะไม่สามารถแก้ไขได้หลังจากการประกาศ จึงต้องทำการประกาศตัวแปรพร้อมระบุค่า
\begin{lstlisting}
const int maxNumber = 100;
maxNumber = 10;	//error: can\'t change constant variable's value
\end{lstlisting}

\newpage
\subsubsection{General rules for variables naming}
\begin{enumerate}
\item ชื่อตัวแปรสามารถมีตัวอักษร, ตัวเลข, และ underscore (\texttt{\_}) เท่านั้น ไม่สามารถมีช่องว่างหรือตัวอักษรพิเศษ เช่น \texttt{!, \#, \%} เป็นต้น
\item ชื่อตัวแปรต้องขึ้นต้นด้วยตัวอักษรหรือ underscore เท่านั้น ไม่สามารถขึ้นต้นด้วยตัวเลขได้
\item ชื่อตัวแปรเป็น case sensitive กล่าวคือตัวอักษรพิมพ์ใหญ่และพิมพ์เล็กมีผลต่อการระบุชื่อตัวแปร (ตัวแปรชื่อ firstNumber และ FirstNumber ถือว่าไม่เป็นตัวแปรเดียวกัน)
\item ชื่อตัวแปรไม่สามารถเป็นคำสงวน (Reserved words) ได้
\end{enumerate}
\subsubsection{Reserved words}
\begin{center}
\begin{tabular}{||c|c|c||}
\hline
alignas & alignof & and \\
and\_eq & asm & atomic\_cancel \\
atomic\_commit & atomic\_noexcept & auto \\
bitand & bitor & bool \\
break & case & catch \\
char & char8\_t & char16\_t \\
char32\_t & class & compl \\
concept & const & consteval \\
constexpr & constinit & const\_cast \\
continue & co\_await & co\_return \\
co\_yield & decltype & default \\
delete & do & double \\
dynamic\_cast & else & enum \\
explicit & export & extern \\
false & float & for \\
friend & goto & if \\
inline & int & long \\
mutable & namespace & new \\
noexcept & not & not\_eq \\
nullptr & operator & or \\
or\_eq & private & protected \\
public & reflexpr & register \\
reinterpret\_cast & requires & return \\
short & signed & sizeof \\
static & static\_assert & static\_cast \\
struct & switch & stnchronized \\
template & this & thread\_local \\
throw & true & try \\
typedef & typeid & typename \\
union & unsigned & using \\
virtual & void & volatile \\
wchar\_t & while & xor \\
xor\_eq &  &  \\
\hline
\end{tabular}
\end{center}

\section{Input/Output}

การรับค่าจากผู้ใช้ และการแสดงผลทางหน้าจอสามารถทำได้แตกต่างกันตามภาษา C หรือ C++
\begin{lstlisting}
int w;
double x;
char y;
char[10] z;

//C language
scanf("%d %lf %c %s", &w, &x, &y, z);
printf("%d %lf %c %s\n", w, x, y, z);
\end{lstlisting}
การรับค่าของภาษา C จะใช้คำสั่ง \textbf{\texttt{scanf}} โดยจะต้องระบุประเภทของตัวแปรที่ต้องการจะรับเข้ามาด้วย ในตัวอย่างจะเห็นได้ว่า
\begin{itemize}
\item ตัวระบุประเภทตัวที่ 1 คือ \texttt{\%d} ซึ่งจะจับคู่กับตัวแปรชื่อ \texttt{w}
\item ตัวระบุประเภทตัวที่ 2 คือ \texttt{\%lf} ซึ่งจะจับคู่กับตัวแปรชื่อ \texttt{x}
\item ตัวระบุประเภทตัวที่ 3 คือ \texttt{\%c} ซึ่งจะจับคู่กับตัวแปรชื่อ \texttt{y}
\item ตัวระบุประเภทตัวที่ 4 คือ \texttt{\%s} ซึ่งจะจับคู่กับตัวแปรชื่อ \texttt{z}
\end{itemize}
และจะสังเกตได้ว่าการจับคู่ในการรับค่าจะต้องใส่เครื่องหมาย \textbf{\texttt{\&}} หน้าตัวแปรทุกตัว นอกจากตัวแปรประเภท char[]
\begin{center}
\begin{tabular}{||c|c||}
\hline
ประเภทตัวแปร & ตัวระบุประเภท \\
\hline
\texttt{int} & \texttt{\%d} \\
\texttt{double} & \texttt{\%lf} \\
\texttt{char} & \texttt{\%c} \\
\texttt{char[]} & \texttt{\%s} \\
\hline
\end{tabular}
\end{center}
การแสดงผลของภาษา C จะใช้คำสั่ง \textbf{\texttt{printf}} โดยจะมีลักษณะคล้ายกับการรับค่าด้วยคำสั่ง \textbf{\texttt{scanf}} กล่าวคือต้องใช้ตัวระบุประเภทเช่นเดียวกัน แต่มีความแตกต่างกันที่ ไม่ต้องใส่เครื่องหมาย \textbf{\texttt{\&}} หน้าตัวแปรใดๆ เลย

\noindent\textbf{หมายเหตุ:} \textbf{\texttt{\textbackslash n}} ที่อยู่ต่อท้ายเป็นการระบุว่า ให้ขึ้นบรรทัดใหม่เมื่อแสดงผลบรรทัดนี้แล้ว

\begin{lstlisting}
//C++ language
cin >> w >> x >> y >> z;
cout << w << ' ' << x << ' ' << y << ' ' << z << endl;
\end{lstlisting}
การรับค่าของภาษา C++ จะใช้คำสั่ง \textbf{\texttt{cin}} โดยมีเครื่องหมาย \textbf{\texttt{>>}} คั่นอยู่ระหว่างคำสั่ง และตัวแปรต่างๆ \\
การแสดงผลของภาษา C++ จะใช้คำสั่ง \textbf{\texttt{cout}} โดยมีเครื่องหมาย \textbf{\texttt{<<}} คั่นอยู่ระหว่างคำสั่ง และตัวแปรต่างๆ \\
นอกจากนี้ โปรแกรมภาษา C++ ก็สามารถใช้คำสั่ง \textbf{\texttt{scanf}} และ \textbf{\texttt{printf}} ของภาษา C ได้เช่นกัน

\noindent\textbf{หมายเหตุ:} \textbf{\texttt{endl}} ที่อยู่ต่อท้ายเป็นการระบุว่า ให้ขึ้นบรรทัดใหม่เมื่อแสดงผลบรรทัดนี้แล้ว (เช่นเดียวกันกับ \texttt{\textbackslash n})

\newpage
\subsection{Operators}
\subsubsection{Arithmetic}
\begin{center}
\begin{tabular}{||c|c|c|c||}
\hline
\textbf{ตัวดำเนินการ} & \textbf{ชื่อ} & \textbf{คำอธิบาย} & \textbf{การใช้งาน} \\
\hline
\texttt{+} & การบวก & บวก 2 จำนวนเข้าด้วยกัน & \texttt{x + y} \\
\texttt{-} & การลบ & ลบ 2 จำนวนเข้าด้วยกัน & \texttt{x - y} \\
\texttt{*} & การคูณ & คูณ 2 จำนวนเข้าด้วยกัน & \texttt{x * y} \\
\texttt{/} & การหาร & หารจำนวนแรกด้วยจำนวนที่ 2 & \texttt{x / y} \\
\texttt{\%} & การหารเอาเศษ & เศษจากการหารจำนวนแรกด้วยจำนวนที่ 2 & \texttt{x \% y} \\
\texttt{++} & Increment & เพิ่มค่าตัวแปรขึ้นไป 1 & \texttt{x++} หรือ \texttt{++x} \\
\texttt{--} & Decrement & ลดค่าตัวแปรลงไป 1 & \texttt{x--} หรือ \texttt{--x} \\
\hline
\end{tabular}
\end{center}
\subsubsection{Assignment}
\begin{center}
\begin{tabular}{||c|c|c||}
\hline
\textbf{ตัวดำเนินการ} & \textbf{การใช้งาน} & \textbf{ผลลัพธ์} \\
\hline
= & \texttt{x = 5} & \texttt{x = 5} \\
+= & \texttt{x += 5} & \texttt{x = x + 5} \\
-= & \texttt{x -= 5} & \texttt{x = x - 5} \\
*= & \texttt{x *= 5} & \texttt{x = x * 5} \\
/= & \texttt{x /= 5} & \texttt{x = x / 5} \\
\%= & \texttt{x \%= 5} & \texttt{x = x \% 5} \\
\&= & \texttt{x \&= 5} & \texttt{x = x \& 5} \\
|= & \texttt{x |= 5} & \texttt{x = x | 5} \\
\hline
\end{tabular}
\end{center}
\subsubsection{Comparison}
\begin{center}
\begin{tabular}{||c|c|c||}
\hline
\textbf{ตัวดำเนินการ} & \textbf{ชื่อ} & \textbf{การใช้งาน} \\
\hline
== & เท่ากับ & \texttt{x == y} \\
!= & ไม่เท่ากับ & \texttt{x != y} \\
> & มากกว่า & \texttt{x > y} \\
< & น้อยกว่า & \texttt{x < y} \\
>= & มากกว่าหรือเท่ากับ & \texttt{x >= y} \\
<= & น้อยกว่าหรือเท่ากับ & \texttt{x <= y} \\
\hline
\end{tabular}
\end{center}

\subsubsection{Logical}
\begin{center}
\begin{tabular}{||c|c|c||}
\hline
\textbf{ตัวดำเนินการ} & \textbf{ชื่อ} & \textbf{การใช้งาน} \\
\hline
\texttt{\&\&} & และ (ตรรกศาสตร์) & \texttt{x > 1 \&\& x < 10} \\
\texttt{||} & หรือ (ตรรกศาสตร์) & \texttt{x < 1 || x > 10} \\
\texttt{!} & นิเสธ & \texttt{!(x > 1 \&\& x < 10)} \\
\hline
\end{tabular}
\end{center}

\newpage
\section{Conditions}
เราสามารถเลือกให้โปรแกรมทำบางคำสั่ง เมื่อเงื่อนไขเป็นจริงเท่านั้น โดยการใช้ \textbf{\texttt{if-else}} หรือ \textbf{\texttt{switch}} ได้
\begin{lstlisting}
if (condition1) {
	// code to be executed if the condition1 is true
}else if (condition2) {
	// code to be executed if the condition1 is false and condition2 is true
} else if (condition3) {
	// code to be executed if the condition1,2 is false and condition 3 is true
} else {
	// code to be executed if the condition1,2,3 is false
}
\end{lstlisting}
การเขียนเงื่อนไขโดยใช้ \textbf{\texttt{if}}, \textbf{\texttt{else if}}, และ \textbf{\texttt{else}} จะเป็นที่นิยมเนื่องจากใช้งานได้ง่ายและสะดวก

\begin{lstlisting}
switch (expression) {
	case x:
    	// code to be executed if expression is equal to x
        break;
    case y:
    	// code to be executed if expression is equal to y
        break;
    default:
    	// code to be executed if expression is not equal to any case
}
\end{lstlisting}
การเขียนเงื่อนไขโดยการใช้ \textbf{\texttt{switch}} จะต้องระบุ \texttt{expression}ให้ตรงกับ \texttt{case} พอดี ไม่สามารถเปรียบเทียบน้อยกว่า หรือมากกว่าได้

\noindent\textbf{หมายเหตุ:} ทุก \texttt{case} ของ \texttt{switch} จะต้องมีคำสั่ง \textbf{\texttt{break}} ต่อท้ายอยู่เสมอ

\newpage
\section{Loop}
การวนซ้ำสามารถใช้ในการทำงานซ้ำหลายๆ ครั้งตราบใดที่เงื่อนไขของการวนซ้ำยังเป็นจริงอยู่ ประโยชน์ของการวนซ้ำคือลดระยะเวลาการเขียนลง ลดการเกิด errors และทำให้โปรแกรมอ่านง่ายขึ้น

การวนซ้ำทุกรูปแบบจะมี 3 สิ่งสำคัญได้แก่
\begin{enumerate}
\item จุดเริ่มต้น (initial)
\item เงื่อนไข (condition)
\item การเปลี่ยนแปลง (update)
\end{enumerate}

\begin{lstlisting}
/*
for (initial; condition; update) {
	//block of code to be executed
}
*/
for (int i = 0; i < 5; i++) {
	printf("%d\n", i);
}

/*
initial
while (condition) {
	//block of code to be executed
    update
}
*/
int i = 0;
while (i < 5) {
	printf("%d\n", i);
    i++;
}
\end{lstlisting}
ตัวอย่างโปรแกรมด้านบนเป็นโปรแกรมเพื่อแสดงผลตัวเลข 0 ถึง 4 บรรทัดละ 1 จำนวน โดยการวนซ้ำด้วย \textbf{\texttt{for}} และ \textbf{\texttt{while}} เริ่มต้นที่ \textbf{\texttt{i = 0}} โดยมีเงื่อนไขการวนซ้ำคือ \textbf{\texttt{i < 5}} และมีส่วนการเปลี่ยนแปลงคือ \textbf{\texttt{i++}}
\begin{itemize}
\item ขณะที่ \textbf{\texttt{i = 0}} ซึ่ง \textbf{\texttt{i < 5}} จึงแสดงผลเลข \textbf{\texttt{0}} ออกมา
\item ขณะที่ \textbf{\texttt{i = 1}} ซึ่ง \textbf{\texttt{i < 5}} จึงแสดงผลเลข \textbf{\texttt{1}} ออกมา
\item ขณะที่ \textbf{\texttt{i = 2}} ซึ่ง \textbf{\texttt{i < 5}} จึงแสดงผลเลข \textbf{\texttt{2}} ออกมา
\item ขณะที่ \textbf{\texttt{i = 3}} ซึ่ง \textbf{\texttt{i < 5}} จึงแสดงผลเลข \textbf{\texttt{3}} ออกมา
\item ขณะที่ \textbf{\texttt{i = 4}} ซึ่ง \textbf{\texttt{i < 5}} จึงแสดงผลเลข \textbf{\texttt{4}} ออกมา
\item ขณะที่ \textbf{\texttt{i = 5}} ซึ่ง \textbf{\texttt{i !< 5}} จึงหยุดการวนซ้ำ
\end{itemize}
นอกจากนี้ยังมีการวนซ้ำแบบ \textbf{\texttt{do-while}} ซึ่งจะทำงานก่อน 1 รอบไม่ว่าเงื่อนไขที่กำหนดจะเป็นจริงหรือไม่ แล้วจึงตรวจสอบเงื่อนไขก่อนทำงานรอบถัดไป
\begin{lstlisting}
int i = 5;
do {
	printf("%d\n", i);
}while (i < 5);
\end{lstlisting}
โปรแกรมด้านบนนี้จะแสดงผลเลข 5 ออกมาแล้วจึงหยุดการทำงาน

ในบางกรณี เราจำเป็นต้องหยุดการวนซ้ำกลางคัน หรือข้ามการวนซ้ำบางขั้นตอน จึงมีคำสั่ง \textbf{\texttt{break}} สำหรับหยุดการวนซ้ำนั้น และ \textbf{\texttt{continue}} สำหรับข้ามการวนซ้ำขั้นตอนหนึ่งๆ
\begin{lstlisting}
for (int i = 0; i < 5; i++) {
	if (i == 2) {
    	break;
    }
	printf("%d\n",i);
}
\end{lstlisting}
โปรแกรมด้านบนนี้จะแสดงผลแค่เลข 0 และเลข 1 บรรทัดละ 1 จำนวน
\begin{itemize}
\item ขณะที่ \textbf{\texttt{i = 0}} ซึ่ง \textbf{\texttt{i < 5}} จึงแสดงผลเลข \textbf{\texttt{0}} ออกมา
\item ขณะที่ \textbf{\texttt{i = 1}} ซึ่ง \textbf{\texttt{i < 5}} จึงแสดงผลเลข \textbf{\texttt{1}} ออกมา
\item ขณะที่ \textbf{\texttt{i = 2}} ซึ่ง \textbf{\texttt{i < 5}} แต่ตรงกับเงื่อนไขของคำสั่ง \textbf{\texttt{break}} จึงหยุดการวนซ้ำทันที
\end{itemize}

\begin{lstlisting}
for (int i = 0; i < 5; i++) {
	if (i == 2) {
    	continue;
    }
	printf("%d\n",i);
}
\end{lstlisting}
โปรแกรมด้านบนนี้จะแสดงผลแค่เลข 0, 1, 3, และ 4 บรรทัดละ 1 จำนวน
\begin{itemize}
\item ขณะที่ \textbf{\texttt{i = 0}} ซึ่ง \textbf{\texttt{i < 5}} จึงแสดงผลเลข \textbf{\texttt{0}} ออกมา
\item ขณะที่ \textbf{\texttt{i = 1}} ซึ่ง \textbf{\texttt{i < 5}} จึงแสดงผลเลข \textbf{\texttt{1}} ออกมา
\item ขณะที่ \textbf{\texttt{i = 2}} ซึ่ง \textbf{\texttt{i < 5}} แต่ตรงกับเงื่อนไขของคำสั่ง \textbf{\texttt{continue}} จึงข้ามการวนซ้ำขั้นตอนนี้ไป และไม่มีการแสดงผลเลข \textbf{\texttt{2}} ออกมา
\item ขณะที่ \textbf{\texttt{i = 3}} ซึ่ง \textbf{\texttt{i < 5}} จึงแสดงผลเลข \textbf{\texttt{3}} ออกมา
\item ขณะที่ \textbf{\texttt{i = 4}} ซึ่ง \textbf{\texttt{i < 5}} จึงแสดงผลเลข \textbf{\texttt{4}} ออกมา
\item ขณะที่ \textbf{\texttt{i = 5}} ซึ่ง \textbf{\texttt{i !< 5}} จึงหยุดการวนซ้ำ
\end{itemize}
