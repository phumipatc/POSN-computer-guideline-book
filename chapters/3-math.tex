\chapter{คณิตศาสตร์ที่ควรรู้ (Math for TOI)}
\section{หลักการนับเบื้องต้น}
\subsection{กฎการบวก (Sum rules)}
\begin{theorem}
ถ้าในกระบวนการที่ต้องการนับ ขั้นตอนหนึ่งมีทางเลือกในการสร้างผลลัพธ์อยู่ $k$ แบบ โดยแบบที่ $i$ ($1 \leq i \leq k$) สร้างผลลัพธ์ได้ $n_i$ วิธี กระบวนการดังกล่าวจะสามารถสร้างผลลัพธ์ได้ทั้งหมด
\begin{center}
$n_1+n_2+n_3+...+n_k$
\end{center}
วิธี  
\end{theorem}

พิจารณาตัวอย่างต่อไปนี้
\begin{example}
ถ้าต้องการเดินทางจากเมือง A ไปยังเมือง B โดยมีรถจักรยานยนต์ที่แตกต่างกัน 3 คัน, รถยนต์ที่แตกต่างกัน 8 คัน, และเรือ 5 ลำ จะสามารถเดินทางได้ทั้งหมดกี่วิธี?

ในการเลือกยานพาหนะนั้น สามารถเลือกยานพาหนะชนิดใดก็ได้ โดยหากจะเลือกรถจักรยานยนต์ก็สามารถเลือกได้ 3 วิธี หากจะเลือกรถยนต์ก็สามารถเลือกได้ 8 วิธี หรือหากจะเลือกใช้เรือก็สามารถเลือกได้ 5 วิธี ดังนั้น จำนวนยานพาหนะที่สามารถเลือกได้ทั้งหมดจะเป็น
\begin{center}
$3+8+5=16$
\end{center}
วิธี
\end{example}

\newpage

\subsection{กฎการคูณ (Product rules)}
\begin{theorem}
ถ้ากระบวนการที่ต้องการนับสามารถแบ่งได้เป็น $k$ ขั้นตอนที่ต้องกระทำต่อเนื่องกัน โดยในขั้นตอนที่ $i$ ($1 \leq i \leq k$) มีจำนวนวิธีที่เกิดขึ้นได้ $n_i$ วิธีเสมอไม่ว่าในขั้นตอนอื่นๆ จะเลือกวิธีใดก็ตาม เราจะสร้างผลลัพธ์ได้ทั้งหมด
\begin{center}
$n_1n_2n_3...n_k$
\end{center}
วิธี
\end{theorem}

พิจารณาตัวอย่างต่อไปนี้
\begin{example}
ถ้ามีเสื้อ 3 ตัว กางเกง 5 ตัว ต้องการเลือกเสื้อและกางเกงอย่างละ 1 ตัวมาใส่ จะสามารถเลือกได้กี่วิธี

ปัญหานี้สามารถแบ่งออกเป็น 2 ขั้นตอนได้ คือ
\begin{enumerate}
\item เลือกเสื้อ 1 ตัวจาก 3 ตัว จะได้วิธีการเลือกเสื้อ 3 วิธี
\item เลือกกางเกง 1 ตัวจาก 5 ตัว จะได้วิธีการเลือกกางเกง 5 วิธี
\end{enumerate}
จะเห็นได้ว่าการเลือกเสื้อตัวใด ไม่ส่งผลต่อการเลือกกางเกงเลย โดยมีวิธีการเลือกกางเกง 5 วิธีเช่นเดิมเสมอ การเลือกเสื้อและกางเกงจึงเป็นไปได้ทั้งหมด $5+5+5 = 3 \cdot 5 = 15$
\end{example}

\subsection{การเรียงสับเปลี่ยน (Permutation)}
\begin{example}
มีของที่แตกต่างกันอยู่ $n$ ชิ้น และมีตำแหน่งว่างสำหรับว่างของ $k$ ตำแหน่ง จะสามารถเรียงของที่มีอยู่ลงไปวางได้ทั้งหมดกี่วิธี

ขั้นตอนแรก ทำการเลือกของ 1 ชิ้นมาวางที่ตำแหน่งแรกสุด ซึ่งทำได้ $n$ วิธี หลังจากนั้นจึงเลือกของอีก 1 ชิ้นมาว่างที่ตำแหน่งที่ 2 จะเห็นว่าไม่ว่าจะเลือกของชิ้นใดมาวางที่ตำแหน่งแรก ทางเลือกในกาวางของในตำแหน่งที่ 2 ก็มี $n-1$ วิธีเช่นเดิมเสมอ และในขั้นตอนต่อๆ ไปก็เป็นเช่นเดียวกันจนครบทั้ง $k$ ตำแหน่ง ดังนั้น จำนวนวิธีในการเรียงสับเปลี่ยนของ $n$ ชิ้นลงในช่องว่าง $k$ ตำแหน่งจึงมีค่าเป็น
\begin{center}
$^{n}P_{r}=n(n-1)(n-2)...(n-k+1) = \frac{n!}{(n-k)!}$
\end{center}
\end{example}

\subsection{การจัดหมู่ (Combination)}
\begin{example}
มีคนอยู่ทั้งหมด $n$ แต่ต้องการเลือกตัวแทนออกมา $k$ คน ($k \leq n$) จะมีกลุ่มของตัวแทนได้ทั้งหมดกี่แบบ

หากพิจารณาคล้ายกับการจัดหมู่ โดยมองว่ามีตำแหน่งว่าง $k$ ตำแหน่งจะทำให้ได้จำนวนวิธีทั้งหมดเป็น
\begin{center}
$\frac{n!}{(n-k)!}$
\end{center}
แต่สังเกตว่ากลุ่มของตัวแทนที่เลือกมานั้นไม่ได้สนใจตำแหน่งของการเลือกว่าเลือกใครเข้ามาก่อน สนใจเพียงแค่ว่าคนๆ นั้นได้เป็นตัวแทนหรือไม่ ดังนั้นการนับแบบนี้จึงมีการนับซำ้เกิดขึ้นไป $k!$ ครั้งสำหรับแต่ละกลุ่มตัวแทน ทำให้จำนวนวิธีทั้งหมดต้องหารด้วย $k!$ ได้เป็น
\begin{center}
$^nC_r = \binom{n}{k} = \frac{n!}{k!(n-k)!}$
\end{center}
โดยจะอ่าน $\binom{n}{k}$ ว่า ``$n$ เลือก $k$'' หรือสัมประสิทธิ์ทวินาม (binomial coefficient) ก็ได้เช่นกัน
\end{example}

\subsection{สัมประสิทธิ์ทวินาม (Binomial coefficient)}
\begin{theorem}
$(x+y)^n = \binom{n}{0}x^n + \binom{n}{1}x^{n-1}y + \binom{n}{2}x^{n-2}y^2 + \binom{n}{n}y^n$
\end{theorem}
หรือกล่าวได้ว่า สัมประสิทธิ์ของ $x^{n-k}y^{k}$ ของการกระจาย $(x+y)^n$ มีค่าเท่ากับ $\binom{n}{k}$

ด้วยทฤษฎีบทนี้ สามารถแสดงให้เห็นคุณสมบัติหลายอย่างของสัมประสิทธิ์ทวินามได้ เช่น เมื่อแทน $x=1$ และ $y=1$ จะได้ว่า
\begin{center}
$\binom{n}{0} + \binom{n}{1} + \binom{n}{2} + \binom{n}{n} = 2^n$
\end{center}

\section{เทคนิคการนับ}
\subsection{การเพิ่มเข้า-ตัดออก (Inclusion-Exclusion Principle)}

หลักการนี้มักจะถูกใช้ร่วมกับการพิจารณาโจทย์ที่เกี่ยวข้องกับเซต ในบางครั้งการใช้หลักการนับเบื้องต้นจะทำให้เกิดการนับซำ้ เราจึงต้องลบส่วนที่เกิดการนับซำ้ออก

\begin{example}
มีจำนวนเต็มกี่จำนวนตั้งแต่ 1 ถึง 300 ที่หารด้วย 2 หรือ 7 ลงตัว

พิจารณาปัญหานี้ โดยแบ่งออกเป็น 2 ส่วน ดังนี้
\begin{itemize}
\item จำนวนเต็มที่หารด้วย 2 ลงตัว มีทั้งหมด $\lfloor \frac{300}{2} \rfloor = 150$ จำนวน
\item จำนวนเต็มที่หารด้วย 7 ลงตัว มีทั้งหมด $\lfloor \frac{300}{7} \rfloor = 42$ จำนวน
\end{itemize}

หากใช้หลักการนับเบื้องต้น รวมจำนวนเต็มจากทั้ง 2 ส่วน จะได้จำนวนรวมทั้งหมด 192 จำนวน แต่จะเห็นได้ว่าจำนวนเต็มบางจำนวนถูกนับซำ้จากทั้ง 2 ส่วน เช่น 14, 28, 42 เป็นต้น ดังนั้นจึงต้องลบจำนวนที่ถูกนับซำ้ออกไป
\begin{itemize}
\item จำนวนเต็มที่หารด้วย 14 ลงตัว มีทั้งหมด $\lfloor \frac{300}{14} \rfloor = 21$ จำนวน
\end{itemize}
จำนวนเต็มทั้งหมดตั้งแต่ 1 ถึง 300 ที่หารด้วย 2 หรือ 7 ลงตัวจึงมีทั้งหมด $150+42-21=171$ จำนวน
\end{example}

\newpage

\begin{example}
ในการสำรวจชมรมของนักเรียนห้องหนึ่ง ซึ่งมีนักเรียนทั้งหมด 50 คน พบว่า มีนักเรียนในชมรมคณิต 22 คน ชมรมวิทย์ 16 คน และชมรมภาษา 19 คน นอกจากนี้ ยังมีนักเรียนที่เข้าร่วมทั้งชมรมคณิตและวิทย์ 8 คน ชมรมคณิตและภาษา 7 คน ชมรมวิทย์และภาษา 5 คน สุดท้ายมีนักเรียนที่เข้าร่วมทั้งสามชมรมนี้ 3 คน อยากทราบว่า มีนักเรียนที่คนที่ไม่ได้เข้าร่วมชมรมใดในสามชมรมนี้เลย

การหาจำนวนนักเรียนที่ไม่ได้เข้าร่วมใดในสามชมรมนี้ สามารถคิดได้จากการนำจำนวนนักเรียนทั้งหมดลบด้วยจำนวนนักเรียนที่เข้าร่วมอย่างน้อย 1 ชมรมใน 3 ชมรมนี้
\begin{center}
$|A \cup B \cup C| = |A| + |B| + |C| - |A \cap B| - |A \cap C| - |B \cap C| + |A \cap B \cap C|$
\end{center}
เมื่อกำหนดให้
\begin{itemize}
\item เซต A คือเซตของนักเรียนในชมรมคณิต
\item เซต B คือเซตของนักเรียนในชมรมวิทย์
\item เซต C คือเซตของนักเรียนในชมรมภาษา
\end{itemize}
เมื่อแทนค่าจากโจทย์ลงไป
\begin{center}
$|A \cup B \cup C| = 22 + 16 + 19 - 8 - 7 - 5 + 3 = 40$
\end{center}
ดังนั้น นักเรียนที่ไม่ได้เข้าร่วมชมรมใดในสามชมรมนี้เลย มีทั้งหมด $50-40=10$ คน
\end{example}

\subsection{หลักรังนกพิราบ (Pigeonhole Principle)}

สมมติว่ามีนกพิราบ 5 ตัว และมีรังนกอยู่บนต้นไม้ 4 รัง เนื่องจากจำนวนนกพิราบมีมากกว่าจำนวนรัง เราจึงทราบได้ทันทีว่าจะต้องมี\textbf{อย่างน้อย 1 รัง}ที่มีนกพิราบอยู่มากกว่า 1 ตัว

\begin{theorem}
ถ้ามีกล่องใส่ของทั้งสิ้น $n$ ใบ และมีของมากกว่า $n$ ชิ้น จะต้องมีกล่องอย่างน้อย 1 ใบที่บรรจุของอยู่มากกว่า 1 ชิ้น
\end{theorem}

\begin{example}
หากทำการยิงปืน 50 นัดไปยังกระดานสี่เหลี่ยมจตุรัสกว้างด้านละ 70 เมตร โดยกระสุนที่ยิงนั้นเข้ากรอบของกระดานทุกนัด จงแสดงว่าจะมีกระสุนอย่างน้อย 2 นัดที่อยู่ห่างกันไม่เกิน 15 เมตร

ถ้าทำการขีดเส้นแบ่งกระดานนี้เป็นตาราง 7 แถว 7 หลักแล้ว จะได้กระดานที่ประกอบไปด้วยช่องทั้งหมด 49 ช่อง แต่ละช่องมีความกว้าง 10 เมตร และเนื่องจากจำนวนกระสุนที่ยิงมีมากกว่าจำนวนช่อง จึงสามารถใช้หลักรังนกพิราบเพื่อบอกว่าจะต้องมีกระสุนอย่างน้อย 2 นัดที่อยู่ในช่องเดียวกัน

เมื่อพิจารณาระยะห่างที่มากที่สุดของกระสุน 2 นัดที่อยู่ในช่องเดียวกัน ระยะที่มากที่สุดที่เป็นไปได้คือความยาวเส้นทแยงมุมของช่องนั้น ซึ่งยาว $10\sqrt{2} \approx 14.142$ เมตร ซึ่งไม่เกิน 15 เมตร จึงสามารถสรุปได้ว่าจะต้องมีกระสุนอย่างน้อย 2 นัดที่อยู่ห่างกันไม่เกิน 15 เมตรอย่างแน่นอน  
\end{example}

\newpage

\subsection{ดาวกับเส้นแบ่ง (Star \& Bars)}
\begin{example}
มีลูกบอลที่เหมือนกัน $n$ ลูก ต้องการแจกให้คน $k$ คน \textbf{โดยแต่ละคนต้องได้ลูกบอลอย่างน้อย 1 ลูก} จะสามารถแจกลูกบอลได้ทั้งหมดกี่วิธี

เนื่องจากลูกบอลทั้งหมดนั้นเหมือนกัน การหาคำตอบโดยวิธีการปกติจึงเป็นไปได้ยาก เราจึงใช้วิธีนำลูกบอลทั้งหมดมาวางเรียงเป็นเส้นตรงแล้วขีดเส้นแบ่งระหว่างลูกบอลเพื่อแบ่งลูกบอลเป็น $k$ กลุ่มแทน หลังจากนั้นจึงเริ่มแจกลูกบอลเป็นกลุ่มๆ โดยเริ่มจากกลุ่มทางซ้ายสุดก่อน

จะพบว่าจำนวนวิธีการแจกลูกบอล ก็คือจำนวนวิธีในการขีดเส้นแบ่งของออกเป็น $k$ กลุ่มนั่นเอง การแบ่งของออกเป็น $k$ กลุ่มต้องใช้เส้นแบ่ง $k-1$ เส้น และมีช่องว่างระหว่างลูกบอล $n-1$ ช่องไว้สำหรับขีดเส้นแบ่ง โดยจะต้องไม่ขีดเส้นซ้ำช่องกัน (เนื่องจากต้องการให้แต่ละคนได้ลูกบอลอย่างน้อย 1 ชิ้น) จำนวนวิธีจึงเป็นการเลือกช่องว่าง $k-1$ ช่องจากทั้งหมด $n-1$ ช่อง
\begin{center}
จำนวนวิธีการแจกลูกบอลเท่ากับ $\binom{n-1}{k-1}$ วิธี
\end{center}
\end{example}

\begin{example}
มีลูกบอลที่เหมือนกัน $n$ ลูก ต้องการแจกให้คน $k$ คน \textbf{โดยอาจจะมีบางคนไม่ได้ลูกบอลสักลูกเลยก็ได้} จะสามารถแจกลูกบอลได้ทั้งหมดกี่วิธี

จะเห็นได้ว่าปัญหาคล้ายกับตัวอย่างก่อนหน้า แตกต่างตรงที่คราวนี้สามารถขีดเส้นซำ้ช่องกันได้ ซึ่งทำให้กลุ่มลูกบอลที่อยู่ระหว่างเส้นแบ่งที่อยู่ช่องเดียวกันไม่มีลูกบอลเลยสักลูก

การนับจำนวนวิธีการขีดเส้นแบ่งนี้จึงเหมือนการเรียงสับเปลี่ยนของ $n+k-1$ ชิ้น ประกอบด้วยลูกบอล $n$ ลูกและเส้นแบ่งอีก $k-1$ เส้น โดยที่ลูกบอลทั้งหมดเหมือนกัน และเส้นแบ่งทั้งหมดก็เหมือนกัน
\begin{center}
จำนวนวิธีการแจกลูกบอลเท่ากับ $\frac{(n+k-1)!}{n!(k-1)!} = \binom{n+k-1}{k-1}$ วิธี
\end{center}
\end{example}

\section{การพิสูจน์}

การพิสูจน์เป็นสิ่งที่สำคัญอย่างยิ่งในการสร้างความจริงทางคณิตศาสตร์ เพื่อยืนยันข้อเท็จจริงที่ถูกค้นพบขึ้นมาใหม่ ให้กลายเป็น ``ทฤษฎีบท (Theorem)'' ที่ถูกรองรับด้วย ``บทพิสูจน์ (Proof)'' ซึ่งการพิสูจน์ก็สามารถทำได้หลายวิธีเช่นกัน

\subsection{การพิสูจน์ตรง (Direct proof)}

การพิสูจน์ตรงเป็นการแสดงเหตุผลแบบตรงๆ เพื่อพิสูจน์ความจริงในรูป $p \implies q$ เป็นส่วนมาก โดยการสมมติให้ $p$ เป็นจริงแล้วใช้เหตุผลเพื่อสรุปว่า $q$ ก็ต้องเป็นจริงเท่านั้น ไม่สามารถเป็นเท็จในขณะที่ $p$ เป็นจริงได้

\begin{theorem}
ผลบวกของจำนวนเต็มคู่ 2 จำนวน เป็นจำนวนเต็มคู่
\end{theorem}

กำหนดให้ $P(a,b)$ แทนข้อความว่า ``$a$ และ $b$ เป็นจำนวนเต็มคู่'' และ $Q(a,b)$ แทนข้อความว่า ``$a + b$ เป็นจำนวนเต็มคู่'' จะทำให้สามารถเขียนทฤษฎีบทนี้ได้เป็น $\forall a \forall b(P(a,b) \implies Q(a,b))$ และใช้นิยามของจำนวนเต็มคู่ที่กล่าวว่า จำนวนเต็มคู่คือจำนวนเต็มทีทมี 2 เป็นตัวประกอบ

\begin{proof}
ให้ $a$ และ $b$ เป็นจำนวนเต็มคู่ จึงสามารถกล่าวได้ว่า $a = 2m$ และ $b = 2n$ สำหรับจำนวนเต็ม $m$ และ $n$ บางจำนวน

\begin{center}
$a+b = 2m + 2n = 2(m+n)$
\end{center}
สังเกตเห็นได้ว่าผลบวกของ $a$ และ $b$ เป็นจำนวนเต็มที่มี 2 เป็นตัวประกอบอยู่ด้วย จึงสรุปได้ว่าผลบวกดังกล่าวเป็นจำนวนเต็มคู่ด้วย
\end{proof}

\subsection{การพิสูจน์ด้วยประพจน์แย้งสลับที่ (Proof by contrapositive)}

การพิสูจน์ด้วยประพจน์แย่งสลับที่เป็นการพิสูจน์ข้อความ $p \implies q$ โดยการแสดงให้เห็นว่า $\neg q \implies \neg p$ เป็นจริง เนื่องจากทั้ง 2 ข้อความนี้สมมูลกัน โดยเริ่มต้นจากการให้ $\neg q$ เป็นจริง และใช้เหตุผลต่างๆ เพื่อสรุปว่า $\neg p$ ต้องเป็นจริงตามไปด้วย

\begin{theorem}
ให้ $a$ เป็นจำนวนเต็มใดๆ ถ้า $a^2$ เป็นจำนวนเต็มคู่ แสดงว่า $a$ ต้องเป็นจำนวนเต็มคู่ด้วย
\end{theorem}

กำหนดให้ $P(a)$ แทนข้อความ ``$a^2$ เป็นจำนวนเต็มคู่'' และ $Q(a)$ แทนข้อความ ``$a$ เป็นจำนวนเต็มคู่'' จะทำให้สามารถเขียนทฤษฎีบทนี้ได้เป็น $\forall a(P(a) \implies Q(a))$

\begin{proof}
การพิสูจน์นี้ ต้องแสดงให้เห็นว่า ถ้า $a$ เป็นจำนวนเต็มคี่ ($\neg Q(a)$) แล้ว $a^2$ ก็จะต้องเป็นจำนวนเต็มคี่ ($\neg P(a)$) ด้วยเช่นกัน

เนื่องจาก $a$ เป็นจำนวนเต็มคี่ แสดงว่ามีจำนวนเต็ม  $n$ ที่ทำให้ $a=2n+1$
\begin{align*}
a^2 &= (2n+1)^2 \\
&= 4n^2+2n+1 \\
&= 2(2n^2+n)+1 \\
&= 2m+1
\end{align*}
เนื่องจาก $n$ เป็นจำนวนเต็ม ดังนั้น $2n^2+n$ ก็ต้องเป็นจำนวนเต็มด้วยเช่นกัน จึงได้ว่า   $a^2$  สามารถเขียนอยู่ในรูป $2m+1$ ได้เมื่อ $m$ เป็นจำนวนเต็ม แสดงว่า $a^2$ ต้องเป็นจำนวนเต็มคี่ด้วยเช่นกัน

เราจึงได้ว่า ถ้า $a$ เป็นจำนวนเต็มคี่แล้ว $a^2$ ก็เป็นจำนวนเต็มคี่ด้วยเช่นกัน ซึ่งสมมูลกับข้อความที่ว่า ถ้า $a^2$ เป็นจำนวนเต็มคู่ แสดงว่า $a$ ต้องเป็นจำนวนเต็มคู่ด้วย
\end{proof}

\subsection{การพิสูจน์โดยการแยกกรณี (Proof by cases)}

ในการพิสูจน์บางครั้ง ก็มีเงื่อนไขที่เป็นไปได้หลายกรณี ซึ่งไม่สามารถพิสูจน์พร้อมกันทั้งหมดได้ จึงต้องแยกพิสูจน์ให้เห็นว่า ทฤษฎีบทเป็นจริงในทุกกรณี

\begin{theorem}
ไม่มีจำนวนเต็มบวก $n$ ที่ทำให้ $n^2+n^3=100$
\end{theorem}

\begin{proof}
สังเกตได้ว่า ถ้า $n>4$ แล้ว $n^3>100$  จึงได้ว่า จำนวนเต็มบวกที่อาจจะทำให้สมการเป็นจริงได้ต้องมีค่าไม่เกิน 4

เมื่อได้ขอบเขตของ $n$ มาแล้ว จึงต้องแยกกรณีเพื่อพิจารณาเมื่อ $n$ มีค่าเป็น 1, 2, 3, และ 4 ตามลำดับ จะเห็นได้ว่า
\begin{itemize}
\item $1^2 + 1^3 = 1+1 = 2 \neq 100$
\item $2^2 + 2^3 = 4+8 = 12 \neq 100$
\item $3^2 + 3^3 = 9+27 = 36 \neq 100$
\item $4^2 + 4^3 = 16+64 = 80 \neq 100$
\end{itemize}
เราจึงสรุปได้ว่า ไม่มีจำนวนเต็มบวก $n$ ใดๆ ที่ $n^2+n^3=100$ เลย
\end{proof}

\subsection{การพิสูจน์โดยข้อขัดแย้ง (Proof by contradiction)}

การพิสูจน์โดยข้อขัดแย้งสามารถพบได้บ่อยมาก โดยการพิสูจน์นี้เกิดมาจากแนวคิดว่า หากข้อความหนึ่งเป็นจริงแล้ว จะทำให้เกิดข้อขัดแย้งที่เป็นไปไม่ได้ แสดงว่าข้อความนั้นไม่สามารถเป็นจริงได้

ขั้นตอนการพิสูจน์ จึงมีดังนี้
\begin{enumerate}
\item สมมติให้ทฤษฎีบทเป็นเท็จ
\item แสดงให้เห็นว่า เมื่อทฤษฎีบทเป็นเท็จ จะทำให้เกิดข้อขัดแย้งที่เป็นไปไม่ได้
\item สรุปจากทั้ง 2 ขั้นตอนว่า ทฤษฎีบทไม่สามารถเป็นเท็จได้ กล่าวคือ ทฤษฎีบทจะต้องเป็นจริง
\end{enumerate}

\begin{theorem}
$\sqrt{2}$ เป็นจำนวนอตรรกยะ
\end{theorem}

\begin{proof}
สมมติให้ $\sqrt{2}$ เป็นจำนวนตรรกยะ แสดงว่าต้องมีจำนวนเต็ม $a$ และ $b$ ที่
\begin{center}
$\frac{a}{b}=\sqrt{2}$
\end{center}
โดย $\frac{a}{b}$ เป็นเศษส่วนอย่างตำ่ เมื่อยกกำลังสองทั้งสองข้าง และทำการย้ายข้าง จะได้ว่า
\begin{center}
$a^2=2b^2$
\end{center}
ซึ่งจะเห็นได้ว่า $a^2$ เป็นจำนวนเต็มคู่

จากทฤษฎีบท 3.3.2 ``ให้ $a$ เป็นจำนวนเต็มใดๆ ถ้า $a^2$ เป็นจำนวนเต็มคู่ แสดงว่า $a$ ต้องเป็นจำนวนเต็มคู่ด้วย'' จึงสามารถเขียน $a$  ในรูปของ $2n$ เมื่อ $n$ เป็นจำนวนเต็มได้
\begin{align*}
2b^2 &= (2n)^2 \\
b^2 &= 2n^2
\end{align*}
จะเห็นได้ว่า $b^2$ เป็นจำนวนเต็มคู่ ดังนั้น $b$ ก็เป็นจำนวนเต็มคู่ด้วยเช่นกัน

เนื่องจาก $a$ และ $b$ เป็นจำนวนเต็มคู่ แสดงว่าทั้ง $a$ และ $b$ มี 2 เป็นตัวประกอบร่วม $\frac{a}{b}$  ข้อสมมติที่บอกว่า $\frac{a}{b}$ เป็นเศษส่วนอย่างตำ่จึงเป็นไปไม่ได้ สรุปได้ว่า $\sqrt{2}$ ไม่สามารถเป็นจำนวนตรรกยะได้
\end{proof}

\section{การอุปนัยเชิงคณิตศาสตร์ (Mathematical Induction)}
\subsection{อุปนัยทั่วไป}

หลักแห่งอุปนัยเชิงคณิตศาสตร์ เป็นเทคนิคที่ใช้ในการเหนี่ยวนำความจริงต่อๆ กันไป สมมติว่าต้องการพิสูจน์ข้อความ $P(n)$ ใดๆ ว่าเป็นจริงสำหรับจำนวนเต็ม $n \geq 0$
\begin{enumerate}
\item $P(0)$ เป็นจริง
\item $\forall n \geq 0 (P(n) \implies P(n+1))$
\end{enumerate}

หากสามารถแสดงความจริงของทั้ง 2 ขั้นตอนนี้ได้ ก็จะสามารถสรุปได้ทันทีว่า $P(n)$ เป็นจริงสำหรับจำนวนเต็ม $n \geq 0$

ขั้นตอนแรก คือขั้นตอนที่แสดงให้เห็นว่า $P(0)$ เป็นจริง ชั้นตอนนี้เรียกว่า ขั้นตอนพื้นฐาน (Basis step) ซึ่งเป็นการพิสูจน์กรณีพื้นฐานของข้อความใดๆ (ไม่จำเป็นต้องพิสูจน์ $P(0)$ เสมอไป อาจจะต้องพิสูจน์ว่า $P(c)$ เป็นจริงแทนได้ หากต้องการพิสูจน์ว่าข้อความนั้นๆ เป็นจริงตั้งแต่จำนวนเต็ม $c$ เป็นต้นไป)

ชั้นตอนที่ 2 คือ ขั้นตอนอุปนัย (Inductive step) ในขั้นตอนนี้ต้องแสดงความเชื่อมโยงว่า ถ้า $P(n)$ เป็นจริง $P(n+1)$ ก็จะเป็นจริงด้วย ซึ่งทำได้โดยให้ $P(n)$ เป็นจริงก่อนแล้วจึงให้เหตุผลว่า $P(n+1)$ เป็นจริงตามไปด้วย ข้อสมมติที่ให้ $P(n)$  เป็นจริง เรียกว่า สมมติฐานอุปนัย (Induction hypothesis)

\begin{theorem}
สำหรับจำนวนเต็ม $k \geq 0$ ใดๆ $\sum_{i=0}^k i = \frac{k(k+1)}{2}$
\end{theorem}
\begin{proof}
กำหนดให้ $P(k)$ แทนข้อความว่า $\sum_{i=0}^k i = \frac{k(k+1)}{2}$
\begin{itemize}
\item Basis step: เมื่อ $k=0$ จะเห็นว่า $\sum_{i=0}^0 i = 0 = \frac{k(k+1)}{2}$ เป็นจริง เราจึงได้ว่า $P(0)$  เป็นจริง
\item Induction hypothesis: สมมติให้ $P(k)$ เป็นจริง คือ $\sum_{i=0}^k i = \frac{k(k+1)}{2}$ เป็นจริง
\item Inductive step: พิจารณา $P(k+1)$ ซึ่งคือ $\sum_{i=0}^{k+1} i = \frac{(k+1)(k+2)}{2}$
\begin{align*}
\sum_{i=0}^{k+1} i &= \sum_{i=0}^{k} i + (k+1) \\
&= \frac{k(k+1)}{2} + (k+1) \\
&= (k+1)(\frac{k}{2} + 1) \\
&=\frac{(k+1)(k+2)}{2}
\end{align*}
\end{itemize}

จะเห็นได้ว่า ถ้า $P(k)$ เป็นจริง เราจะได้ว่า $P(k+1)$ เป็นจริงตามไปด้วย จากหลักแห่งอุปนัยเชิงคณิตศาสตร์ สรุปได้ว่า $\sum_{i=0}^k i = \frac{k(k+1)}{2}$ สำหรับจำนวนเต็ม $k \geq 0$ ใดๆ
\end{proof}

\begin{definition}
สำหรับจำนวนเต็ม $a$ และ $b$ ใดๆ เราจะกล่าวว่า $a$ หาร $b$ ลงตัว (หรือ $b$ หารด้วย $a$ ลงตัว) เขียนแทนด้วย $a|b$ ก็ต่อเมื่อมีจำนวนเต็ม $q$ ที่ $b=aq$
\end{definition}
\begin{theorem}
สำหรับจำนวนธรรมชาติ $n$ ใดๆ $n^3-n$ หารด้วย 3 ลงตัว
\end{theorem}
\begin{proof}
กำหนดให้ $P(n)$ แทนข้อความว่า $n^3-n$ หารด้วย 3 ลงตัว
\begin{itemize}
\item Basis step: $P(0)$ คือ $3|(0^3-0)$ หรือ $3|0$ ซึ่งเป็นจริง เนื่องจาก $0=3 \cdot 0$
\item Induction hypothesis: ให้ $P(n)$ เป็นจริง คือ $3|(n^3-n)$
\item inductive step: พิจารณา $P(n+1)$ ซึ่งคือ $3|((n+1)^3-(n+1))$ และจากอุปนัยเชิงคณิตศาสตร์ที่ให้ $3|(n^3-n)$ แสดงว่าจะต้องมีจำนวนเต็ม $q$ ที่ $(n^3-n)=3q$
\begin{align*}
(n+1)^3-(n+1) &= (n^3+3n^2+3n+1)-(n+1) \\
&= (n^3-n)+3n^2+3n \\
&= 3q+3n^2+3n \\
&= 3(q+n^2+n)
\end{align*}
\end{itemize}
ซึ่งเห็นได้ว่า $(n+1)^3-(n+1)$ มี 3 เป็นตัวประกอบ ดังนั้นจึงต้องหารด้วย 3 ลงตัว และจากหลักอุปนัยเชิงคณิตศาสตร์ เราจึงได้ว่า $n^3-n$ หารด้วย 3 ลงตัว เมื่อ $n$ เป็นจำนวนธรรมชาติใดๆ
\end{proof}

\begin{theorem}
สำหรับจำนวนธรรมชาติ $n > 1$ ใดๆ จะได้ว่า $n! < n^n$
\end{theorem}
\begin{proof}
กำหนดให้ $P(n)$ แทนข้อความว่า $n! < n^n$
\begin{itemize}
\item Basis step: $P(2)$ คือ $2!=2 \cdot 1 < 4 = 2^2$ ซึ่งเป็นจริง
\item Induction hypothesis: ให้ $P(n)$ เป็นจริง คือ $n! < n^n$
\item inductive step: พิจารณา $P(n+1)$ ซึ่งคือ $(n+1)! < (n+1)^{n+1}$
\begin{align*}
(n+1)! &= (n+1) \cdot n! \\
&< (n+1) \cdot n^n \\
&< (n+1) \cdot (n+1)^n \\
&= (n+1)^{n+1}
\end{align*}
\end{itemize}
จึงได้ว่า $n! < n^n$ สำหรับจำนวนเต็ม $n > 1$
\end{proof}

\subsection{อุปนัยแบบเข้ม (Strong Induction)}

การอุปนัยแบบเข้มจะคล้ายกับการอุปนัยทั่วไป แตกต่างกันตรงที่การอ้างสมมติฐานของอุปนัยแบบเข้มจะสมมติให้ $P(i)$ เป็นจริงสำหรับค่า $i$ ทุกค่าตั้งแต่ 1 จนถึง $k$ เพื่อเหนี่ยวนำให้ $P(k+1)$ เป็นจริง
\begin{enumerate}
\item $P(1)$ เป็นจริง
\item $\forall n \geq 1 ((P(1) \wedge P(2) \wedge P(3) \wedge ... \wedge P(n)) \implies P(n+1))$
\end{enumerate}

\begin{theorem}
จำนวนเต็มทุกตัวที่มากกว่า 1 สามารถเขียนในรูปผลคูณของจำนวนเฉพาะได้
\end{theorem}
\begin{proof}
กำหนดให้ $P(n)$ แทนข้อความว่า จำนวนเต็ม $n$ สามารถเขียนในรูปผลคูณของจำนวนเฉพาะได้
\begin{itemize}
\item Basis step: พิจารณาจำนวนเต็ม 2 เนื่องจาก 2 เป็นจำนวนเฉพาะอยู่แล้ว จึงสามารถเขียนในรูปผลคูณของจำนวนเฉพาะได้
\item Induction hypothesis: ให้ $P(k)$ เมื่อ $1 < k \leq n$ เป็นจริง คือ จำนวนเต็ม $k$ ที่ $1 < k \leq n$ สามารถเขียนในรูปของจำนวนเฉพาะได้ทั้งหมด
\item inductive step: พิจารณา $P(n+1)$ ซึ่งสามารถแยกออกเป็น 2 กรณี คือ
\begin{enumerate}
\item ถ้า $n+1$ เป็นจำนวนเฉพาะ ก็จะได้ว่า $n+1$ สามารถเขียนในรูปผลคูณของจำนวนเฉพาะได้
\item ถ้า $n+1$ ไม่ใช่จำนวนเฉพาะ แสดงว่าต้องมีจำนวนเต็ม $a$ และ $b$ ที่ $1 < a,b < n+1$ ที่ $ab=n+1$ และจากสมมติฐานอุปนัย เราจะได้ว่าทั้ง $a$ และ $b$ จะต้องสามารถเขียนในรูปของผลคูณของจำนวนเฉพาะได้ ดังนั้น ถ้าเอาผลคุณของจำนวนเฉพาะของ $a$ และ $b$ มาคูณกัน ก็จะได้ว่าสามารถเขียน $n+1$ ในร๔ปของจำนวนเฉพาะได้เช่นกัน
\end{enumerate}
\end{itemize}
จากหลักแห่งอุปนัย จึงสรุปได้ว่า จำนวนเต็มทุกตัวที่มากกว่า 1 สามารถเขียนในรูปผลคูณของจำนวนเฉพาะได้
\end{proof}
\section*{แบบฝึกหัด}
\subsection*{หลักการนับเบื้องต้น}
\begin{exercise}
ให้ $A={a,b,c,d,e,f}$ และ $B={c,d,e}$ จงหาจำนวนซับเซตของ $A$ ที่มีสมาชิกร่วมกับ $B$  หนึ่งตัวพอดี
\end{exercise}
\begin{exercise}
จงหาจำนวนตัวเลขสี่หลัก ตามเงื่อนไขต่อไปนี้
\begin{enumerate}
\item ไม่มีตัวเลขซำ้กัน
\item เป็นเลขคู่
\end{enumerate}
\end{exercise}
\begin{exercise}
สามารถสลับลำดับอักษรจากคำว่า `BIGBARREL'' ให้ต่างกันได้กี่แบบ
\end{exercise}
\begin{exercise}
จงนับจำนวนหมายเลขโทรศัพท์มือถือ ซึ่งประกอบไปด้วยตัวเลข 10 ตัว และต้องขึ้นต้นด้วย 08- หรือ 09- ว่ามีทั้งหมดกี่หมายเลข
\end{exercise}
\begin{exercise}
จงพิสูจน์ว่า
\begin{center}
$1+2\binom{n}{1}+4\binom{n}{2}+...+2^n\binom{n}{n} = 3^n$
\end{center}
\end{exercise}
\begin{exercise}
จงหาจำนวนวิธีการแบ่งของแตกต่างกัน 24 ชิ้นให้เด็ก 4 คน คนละ 6 ชิ้นพอดี
\end{exercise}

\subsection*{เทคนิคการนับ}
\begin{exercise}
ในเมืองหนึ่งมีฟาร์มอยู่ทั้งสิ้น 350 แห่ง ประกอบไปด้วยฟาร์มบลูเบอร์รี่ 260 แห่ง, ฟาร์มมะม่วง 100 แห่ง, ฟาร์มหัวไชเท้า 40 แห่ง, ฟาร์มผสมบลูเบอร์รี่กับหัวไชเท้า 40 แห่ง, ฟาร์มผสมมะม่วงกับหัวไชเท้า 40 แห่ง, และฟาร์มผสมบลูเบอร์รี่กับมะม่วง 30 แห่ง จงหาจำนวนฟาร์มผสมบลูเบอร์รี่, มะม่วง, และหัวไชเท้า
\end{exercise}
\begin{exercise}
จงแสดงว่า ถ้าเลือกจำนวนเต็มสิบตัวจากเซต ${1,2,3,...,17}$ จะต้องมีคู่หนึ่งที่ผลรวมเท่ากับ 18
\end{exercise}
\begin{exercise}
จงแสดงว่า ถ้าเลือกจำนวนเต็มบวกที่เป็นจำนวนคู่และมีค่าน้อยกว่า 1000 มา 38 จำนวน จะต้องมีอย่างน้อยหนึ่งคู่ที่มีผลต่างไม่เกิน 26
\end{exercise}
\begin{exercise}
มีถุงเท้าสีดำ 6 คู่, สีเขียว 4 คู่, สีขาวและสีแดงอย่างละ 5 คู่ รวมกันอยู่ในกล่อง จะต้องหยิบถุงเท้าทีละข้างออกจากกล่องน้อยที่สุดกี่ครั้ง จึงมั่นใจได้ว่าจะได้ถุงเท้าสีเดียวกันอย่างน้อยหนึ่งคู่
\end{exercise}
\begin{exercise}
จงหาจำนวนวิธีระบายสีของ $n$ ชิ้นโดยใช้สี 3 สี เมื่อกำหนดให้ทุกสีต้องใช้ระบายของอย่างน้อยหนึ่งชิ้น และของทุกชิ้นเหมือนกันทั้งหมด
\end{exercise}
\begin{exercise}
จงหาจำนวนวิธีในการแจกเหรียญ $n$ เหรียญให้เด็ก $k$ คน โดยกำหนดให้เด็กทุกคนต้องได้อย่างน้อย 3 เหรียญ
\end{exercise}
\begin{exercise}
จงหาจำนวนเซตของคำตอบที่เป็นจำนวนเต็มไม่ลบของสมการนี้
\begin{center}
$x+y+z=12$
\end{center}
\end{exercise}
\begin{exercise}
มีหมายเลขโทรศัพท์ทั้งหมดกี่หมายเลขที่ประกอบไปด้วยตัวเลข 7 หลัก ซึ่งเป็นลำดับไม่เพิ่ม (ลำดับไม่เพิ่ม คือลำดับที่ตัวเลขหลักใดๆ มีค่าน้อยกว่าหรือเท่ากับตัวเลขในหลักก่อนหน้า)
\end{exercise}

\subsection*{การพิสูจน์}
\begin{exercise}
จงพิสูจน์ว่า ถ้า $n$ เป็นจำนวนเต็มบวก $n$ จะเป็นจำนวนคี่ก็ต่อเมื่อ $5n+6$ เป็นจำนวนเต็มคี่
\end{exercise}
\begin{exercise}
จงพิสูจน์ว่า ถ้า $n$ เป็นจำนวนเต็มที่ $n^3+5$ เป็นจำนวนคี่ $n$ จะต้องเป็นจำนวนคู่
\end{exercise}
\begin{exercise}
จงพิสูจน์ว่าผลคูณของจำนวนตรรกยะสองตัว จะเป็นจำนวนตรรกยะ
\end{exercise}
\begin{exercise}
จงพิสูจน์ว่า ถ้า $p$ เป็นจำนวนอตรรกยะ และ $q$ เป็นจำนวนตรรกยะ จะได้ว่า $pq$ ต้องเป็นจำนวนอตรรกยะ โดยใช้ข้อขัดแย้ง
\end{exercise}
\begin{exercise}
พิจารณาประโยค ``จำนวนเต็มบวกใดๆ สามารถเขียนในรูปผลบวกของกำลังสองของจำนวนเต็มสามตัวได้เสมอ'' เช่น $5=0^2+1^2+2^2$ ประโยคนี้เป็นจริงหรือไม่ หากจริง จงแสดงบทพิสูจน์ หากไม่จริง จงแสดงตัวอย่างค้าน
\end{exercise}
\subsection*{การอุปนัยเชิงคณิตศาสตร์}
\begin{exercise}
จงพิสูจน์ว่า $1+3+5+...+(2n-1)=n^2$ เมื่อ $n \geq 1$
\end{exercise}
\begin{exercise}
จงพิสูจน์ว่า $n^2-1$ หารด้วย 4 ลงตัว เมื่อ $n$ เป็นจำนวนเต็มบวกคี่
\end{exercise}
\begin{exercise}
จงพิสูจน์ว่า $4^n < 1 \cdot 2 \cdot 3 \cdot \cdot \cdot n$ สำหรับจำนวนเต็ม $n>8$
\end{exercise}