\chapter{ฟังก์ชั่น และฟังก์ชั่นเวียนบังเกิด}
ฟังก์ชั่นคือส่วนของโปรแกรมที่จะทำงานก็ต่อเมื่อถูกเรียกใช้งานเท่านั้น ฟังก์ชั่นมักจะถูกใช้เป็นการทำงานเฉพาะๆ หรืองานส่วนหนึ่งๆ ซึ่งมีโอกาสถูกเรียกใช้ซำ้ฟังก์ชั่นจึงสามารถเข้ามาช่วยแก้ปัญหานี้ได้ โดยการสร้างฟังก์ชั่นขึ้นมา แล้วจึงเรียกใช้ฟังก์ชั่นนี้ซำ้นั่นเอง

\section{การสร้างฟังก์ชั่น}
การสร้างฟังก์ชั่นสามารถทำได้โดยการระบุประเภทฟังก์ชั่น, ชื่อฟังก์ชั่น และการทำงานของฟังก์ชั่นนั้นๆ
\begin{lstlisting}
type functionName() {
	//a block of code to be executed when function is called
}
\end{lstlisting}
ประเภทของฟังก์ชั่นใช้สำหรับบอกว่าฟังก์ชั่นนี้จะคืนค่าประเภทใดออกมาหลังจากจบการทำงานของฟังก์ชั่นนี้แล้ว โดยจะเหมือนกันกับประเภทของตัวแปร เพียงแต่มีเพิ่มเติมมา 1 ชนิดที่ควรต้องทราบ คือ \textbf{\texttt{void}} ซึ่งเป็นประเภทของฟังก์ชั่นที่ไม่มีการคืนค่าใดๆ กลับมา
\begin{lstlisting}
void showHelloToTheWorld(){
	printf("Hello, world.\n");
	// no need to return anything
}
int currentYear(){
	return 2022;
	// return some integer
}
\end{lstlisting}

นอกจากนี้ ฟังก์ชั่นสามารถรับค่าพารามิเตอร์ (เปรียบเสมือนข้อมูลนำเข้าสำหรับฟังก์ชั่น) ได้โดยการใส่ประเภทตัวแปร และชื่อตัวแปรลงไปใน \texttt{( )} หลังชื่อฟังก์ชั่น
\begin{lstlisting}
int sum(int a, int b){
	return a + b;
}
\end{lstlisting}

การส่งค่าพารามิเตอร์เข้าไปยังฟังก์ชั่นต่างๆ นั้น เสมือนเป็นการสร้างตัวแปรใหม่ขึ้นมาที่มีค่าเหมือนเดิมทุกประการ เพียงแต่การเปลี่ยนแปลงตัวแปรนั้นภายในฟังก์ชั่นจะไม่ส่งผลกระทบต่อค่าเดิมของตัวแปรที่เป็นต้นแบบในการส่งเข้าไป
\begin{lstlisting}
void add(int a, int b){
	a += b;
}
int main(){
	int a = 10, b = 5;
	add(a, b);
	printf("%d\n", a);
}
\end{lstlisting}
โปรแกรมด้านบนนี้เป็นโปรแกรมเพื่อบวกค่าของตัวแปร \texttt{b} เข้าไปในตัวแปร \texttt{a} แต่หากลองให้โปรแกรมนี้ทำงานแล้ว ผลลัพธ์จะออกมาว่าค่าของตัวแปร \texttt{a} ยังมีค่าเป็น 10 เหมือนเดิมไม่มีการเปลี่ยนแปลงใดๆ เพราะการบวกค่าตัวแปร \texttt{a} ภายในฟังก์ชั่น \texttt{add} เป็นตัวแปรคนละตัวกับตัวแปร \texttt{a} ภายในฟังก์ชั่น \texttt{main} นั่นเอง

เราจึงต้องใช้การ pass by reference เข้ามาช่วย ซึ่งจะเป็นการส่งตัวแปรเข้าไปจริงๆ ไม่ใช่การสร้างตัวแปรใหม่ที่มีค่าเหมือนกัน
\begin{lstlisting}
void add(int &a, int &b){
	a += b;
}
int main(){
	int a = 10, b = 5;
	add(a, b);
	printf("%d\n", a);
}
\end{lstlisting}
ผลลัพธ์จากการทำงานของโปรแกรมด้านบนนี้จะได้ค่าของตัวแปร \texttt{a} ออกมาเป็น 15

\section{ฟังก์ชั่นเวียนบังเกิด (Recursive function)}
การเรียกซ้ำ คือการให้ฟังก์ชั่นเรียกใช้ตัวเอง เพื่อทำการย่อยปัญหาให้เล็กลง แล้วแก้ปัญหาย่อยนั้นก่อนที่จะส่งผลลัพธ์จากการแก้ปัญหาย่อยนั้นกลับไปเพื่อแก้ปัญหาที่ใหญ่ขึ้น โดยจะเรียกฟังก์ชั่นเหล่านี้ว่า ฟังก์ชั่นเวียนบังเกิด (Recursive function)

Recursive function อาจจะดูเป็นเรื่องยากที่จะเข้าใจ แต่หากลองฝึกทำโจทย์บ่อยๆ จะเข้าใจได้มากยิ่งขึ้น และนอกจากนี้ recursive function ยังเป็นพื้นฐานสำหรับเรื่องต่างๆ อีกมากมายที่เราจะได้เรียนภายในค่าย
\begin{lstlisting}
int sum(int k){
	if (k == 0){
		return 0;
	}
	return k + sum(k - 1);
}
\end{lstlisting}
ฟังก์ชั่นด้านบนนี้เป็น recursive function เพื่อหาผลรวมของจำนวนตั้งแต่ 0 ถึง \texttt{k} เช่น หากเมื่อเรียกใช้ \textbf{\texttt{sum(5)}} สิ่งที่เกิดขึ้นจะเป็นขั้นตอน ดังนี้
\begin{itemize}
\item sum(5)
\item 5 + sum(4)
\item 5 + (4 + sum(3))
\item 5 + (4 + (3 + sum(2)))
\item 5 + (4 + (3 + (2 + sum(1))))
\item 5 + (4 + (3 + (2 + (1 + sum(0)))))
\item 5 + (4 + (3 + (2 + (1 + 0))))
\end{itemize}
จะเห็นได้ว่า ฟังก์ชั่น \texttt{sum} ได้ทำการเรียกตัวเองซ้ำ แต่มีการเปลี่ยนแปลงค่าพารามิเตอร์ที่ลดน้อยลงเรื่อยๆ จน \texttt{k = 0} จึงไม่มีการเรียกตัวเองซ้ำอีกแล้ว

\begin{lstlisting}
int fibonacci(int k){
	if (k == 0){
		return 0;
	}
	if (k == 1){
		return 1;
	}
	return fibonacci(k - 2) + fibonacci(k - 1);
}
\end{lstlisting}
ฟังก์ชั่นด้านบนนี้เป็นฟังก์ชั่นหาค่า fibonacci ลำดับที่ \texttt{k} เช่น เมื่อต้องการหาค่า fibonacci ลำดับที่ 4 จะมีขั้นตอนดังนี้
\begin{itemize}
\item \textbf{fibonacci(4)}
\item \textbf{fibonacci(2)} + fibonacci(3);
\item (\textbf{fibonacci(0)} + fibonacci(1)) + fibonacci(3)
\item (0 + \textbf{fibonacci(1)}) + fibonacci(3)
\item (0 + 1) + \textbf{fibonacci(3)}
\item (0 + 1) + (\textbf{fibonacci(1)} + fibonacci(2))
\item (0 + 1) + (1 + \textbf{fibonacci(2)})
\item (0 + 1) + (1 + (\textbf{fibonacci(0)} + fibonacci(1)))
\item (0 + 1) + (1 + (0 + \textbf{fibonacci(1)}))
\item (0 + 1) + (1 + (0 + 1))
\end{itemize}

	Recursive function ที่ยกมาข้างต้นเป็นตัวอย่างพื้นฐานเท่านั้น ยังมีอีกมากมายที่ซับซ้อนมากกว่านี้ เช่น โปรแกรมการเรียงสับเปลี่ยน และการจัดหมู่เป็นต้น
\begin{example}
ฟังก์ชั่นต่อไปนี้เป็นฟังก์ชั่นแสดงผลการเรียงสับเปลี่ยนของตัวเลขตั้งแต่ 1 จนถึง n ออกมา บรรทัดละ 1 ลำดับ
    \begin{lstlisting}
void permutation(int state){
	if (state == n){
		for (int i = 0; i < n; i++){
			printf("%d ", currentPermutation[i]);
		}
		printf("\n");
		return;
	}
	for (int i = 1; i <= n; i++){
		if (!isused[i]){
			isused[i] = true;
			currentPermutation[state] = i;
			permute(state + 1);
			isused[i] = false;
		}
	}
}
    \end{lstlisting}
\end{example}